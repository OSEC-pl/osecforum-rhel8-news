\documentclass[dvipsnames,table]{beamer}
\usepackage{polski}

\usetheme{Rochester}
\usecolortheme{orchid}

\usepackage{listings}
\usepackage{ucs}
\usepackage[utf8x]{inputenc}
\usepackage{wasysym}
\usepackage[normalem]{ulem}
\usepackage{amsmath}
\usepackage{hyperref}
\usepackage{tikzsymbols}

\setbeamertemplate{navigation symbols}{}
\setbeamertemplate{caption}[numbered]
\setbeamerfont{caption}{size=\scriptsize}
\setbeamercolor{framenote}{bg=OSEC-red!25}
\setbeamercolor{rednote}{bg=Red!25}
\setbeamercolor{palette primary}{use=structure,fg=white,bg=OSEC-red}
\setbeamercolor{palette secondary}{use=structure,fg=white,bg=OSEC-red2}

\setbeamertemplate{itemize item}{\scriptsize\raise1pt\hbox{\donotcoloroutermaths$\blacktriangleright$}}
\setbeamertemplate{itemize subitem}{\tiny\raise1pt\hbox{\donotcoloroutermaths$\bullet$}}
\setbeamertemplate{itemize subsubitem}{\tiny\raise1pt\hbox{\donotcoloroutermaths{--}}}

\setbeamertemplate{enumerate item}{\insertenumlabel.}
\setbeamertemplate{enumerate subitem}{\insertenumlabel.\insertsubenumlabel}
\setbeamertemplate{enumerate subsubitem}{\insertenumlabel.\insertsubenumlabel.\insertsubsubenumlabel}
\setbeamertemplate{enumerate mini template}{\insertenumlabel}

\setbeamercolor{itemize item}{fg=OSEC-red, bg=OSEC-red}
\setbeamercolor{itemize subitem}{fg=OSEC-red, bg=OSEC-red}
\setbeamercolor{itemize subsubitem}{fg=OSEC-red, bg=OSEC-red}

\setbeamercolor{section number projected}{fg=white,bg=OSEC-red}
\setbeamercolor{subsection number projected}{fg=white,bg=OSEC-red}
\setbeamercolor{button}{bg=OSEC-red,fg=white}

\setbeamertemplate{section in toc}[circle]
\setbeamertemplate{subsection in toc}[square]

\definecolor{OSEC-red}{RGB}{160,29,44}
\definecolor{OSEC-red2}{RGB}{177,76,12}
\hypersetup{colorlinks=true,linkcolor=white,urlcolor=OSEC-red}

\setlength{\tabcolsep}{8pt}
\renewcommand{\arraystretch}{1.2}

\newcommand{\tri}{$\triangleright$ }

\lstset{
   language=C,
   basicstyle=\tiny,
   breaklines=true,
   escapechar=\@,
}

\title{Nowości w Red Hat Enterprise Linux 8}
\author{Radosław Kujawa -- radoslaw.kujawa@osec.pl}
\institute{OSEC}

\begin{document}

\begin{frame}
	\titlepage
\end{frame}

\begin{frame}
\frametitle{RHEL 8\ldots marketingowo!}
\begin{itemize}
	\item ,,Cloud-native OS''
	\item ,,Brinding hybrid cloud to DevOps''
	\item \href{http://www.redhat.com}{Red Hat}
	\item Fooba {\tt foobar} {\em bardzo}.
\end{itemize}
\begin{center}
\end{center}
\end{frame}

\begin{frame}[fragile]
	\frametitle{Kontenery w RHEL 8 a docker}
\begin{table}[]
\centering
\caption{Docker vs podman compat vs podman native.}
\label{porownanie}
\scriptsize
\begin{tabular}{llll}
\hline
docker & rhel8    \\ \hline
docker run & podman run \\
docker build & buildah  \\
docker foo & bar \\ \hline
\end{tabular}
\normalsize
\end{table}
\end{frame}

\begin{frame}[fragile]
	\frametitle{Przykład: foo}
\begin{lstlisting}
int main(void) {
	return 0;
}
\end{lstlisting}
\end{frame}

\begin{frame}
\frametitle{Koniec\ldots}
\begin{center}
\includegraphics[scale=0.5]{img-oseclogo.png}

Dziękuje!

Czy są pytania?

\end{center}
\end{frame}
\end{document}

