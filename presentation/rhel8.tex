\documentclass[dvipsnames,table]{beamer}
\usepackage{polski}

\usetheme{Rochester}
\usecolortheme{orchid}

\usepackage{listings}
\usepackage{ucs}
\usepackage[utf8x]{inputenc}
\usepackage{wasysym}
\usepackage[normalem]{ulem}
\usepackage{amsmath}
\usepackage{hyperref}
\usepackage{tikzsymbols}

\setbeamertemplate{navigation symbols}{}
\setbeamertemplate{caption}[numbered]
\setbeamerfont{caption}{size=\scriptsize}
\setbeamercolor{framenote}{bg=OSEC-red!25}
\setbeamercolor{rednote}{bg=Red!25}
\setbeamercolor{palette primary}{use=structure,fg=white,bg=OSEC-red}
\setbeamercolor{palette secondary}{use=structure,fg=white,bg=OSEC-red2}

\setbeamertemplate{itemize item}{\scriptsize\raise1pt\hbox{\donotcoloroutermaths$\blacktriangleright$}}
\setbeamertemplate{itemize subitem}{\tiny\raise1pt\hbox{\donotcoloroutermaths$\bullet$}}
\setbeamertemplate{itemize subsubitem}{\tiny\raise1pt\hbox{\donotcoloroutermaths{--}}}

\setbeamertemplate{enumerate item}{\insertenumlabel.}
\setbeamertemplate{enumerate subitem}{\insertenumlabel.\insertsubenumlabel}
\setbeamertemplate{enumerate subsubitem}{\insertenumlabel.\insertsubenumlabel.\insertsubsubenumlabel}
\setbeamertemplate{enumerate mini template}{\insertenumlabel}

\setbeamercolor{itemize item}{fg=OSEC-red, bg=OSEC-red}
\setbeamercolor{itemize subitem}{fg=OSEC-red, bg=OSEC-red}
\setbeamercolor{itemize subsubitem}{fg=OSEC-red, bg=OSEC-red}

\setbeamercolor{section number projected}{fg=white,bg=OSEC-red}
\setbeamercolor{subsection number projected}{fg=white,bg=OSEC-red}
\setbeamercolor{button}{bg=OSEC-red,fg=white}

\setbeamertemplate{section in toc}[circle]
\setbeamertemplate{subsection in toc}[square]

\definecolor{OSEC-red}{RGB}{160,29,44}
\definecolor{OSEC-red2}{RGB}{177,76,12}
\hypersetup{colorlinks=true,linkcolor=white,urlcolor=OSEC-red}

\setlength{\tabcolsep}{8pt}
\renewcommand{\arraystretch}{1.2}

\newcommand{\tri}{$\triangleright$ }


\newcommand\YAMLcolonstyle{\color{red}\mdseries}
\newcommand\YAMLkeystyle{\color{black}\bfseries}
\newcommand\YAMLvaluestyle{\color{black}\mdseries}

\makeatletter

\newcommand\language@yaml{yaml}

\expandafter\expandafter\expandafter\lstdefinelanguage
\expandafter{\language@yaml}
{
  keywords={true,false,null,y,n},
  keywordstyle=\color{darkgray}\bfseries,
  basicstyle=\YAMLkeystyle,                                 % assuming a key comes first
  sensitive=false,
  comment=[l]{\#},
  morecomment=[s]{/*}{*/},
  commentstyle=\color{purple}\ttfamily,
  stringstyle=\YAMLvaluestyle\ttfamily,
  moredelim=[l][\color{orange}]{\&},
  moredelim=[l][\color{magenta}]{*},
  moredelim=**[il][\YAMLcolonstyle{:}\YAMLvaluestyle]{:},   % switch to value style at :
  morestring=[b]',
  morestring=[b]",
  literate =    {---}{{\ProcessThreeDashes}}3
                {>}{{\textcolor{red}\textgreater}}1     
                {|}{{\textcolor{red}\textbar}}1 
                {\ -\ }{{\mdseries\ -\ }}3,
}

% switch to key style at EOL
\lst@AddToHook{EveryLine}{\ifx\lst@language\language@yaml\YAMLkeystyle\fi}
\makeatother

\newcommand\ProcessThreeDashes{\llap{\color{cyan}\mdseries-{-}-}}

\lstdefinestyle{yaml}
{
   language=yaml,
   basicstyle=\small\ttfamily,
   breaklines=true,
   escapechar=\@,
}
\lstdefinestyle{bash}
{
   language=bash,
   basicstyle=\tiny\ttfamily,
   breaklines=true,
   escapechar=\@,
}

\title{Nowości w Red Hat Enterprise Linux 8}
\author{Radosław Kujawa -- radoslaw.kujawa@osec.pl}
\institute{OSEC}

\begin{document}

\begin{frame}
	\titlepage
\end{frame}

\begin{frame}
\frametitle{RHEL 8\ldots marketingowo!}
\begin{itemize}
	\item ,,Cloud-native OS''
	\item ,,Brinding hybrid cloud to DevOps''
	\item \href{http://www.redhat.com}{Red Hat}
	\item Fooba {\tt foobar} {\em bardzo}.
	\item Infrastructure as a Code.
\end{itemize}
\begin{center}
\end{center}
\end{frame}

% cock-pit

% omg IdM w cockpicie

% lorax-composer

% dnf yum
% modularność, application streams - a idea SCL
% zaktualizowane języki programowania - python3, php7 w bazie
% postgresql 10, 9.6, mariadb 10.3, mysql 8, redis 5
% varnish

% nftables, firewalld
% ipvlan w kontenerach

% wayland! gnom 3.28, fallback do X11

% v6 only boot

% authselect, authconfig plz die
%

\begin{frame}[fragile]
	\frametitle{Nowe narzędzia do konteneryzacji}
\begin{table}[]
\begin{itemize}
	\item podman
	\begin{itemize}
		\item Fully supported, not a technology preview!
		\item Daemonless!
		\item User-containers! (Tech prev)
		\item Plays well with systemd!
		\item Red Hat Universal Base Images (UBI) are newly available. No subscription!
		\item ARM!
	\end{itemize}
\end{itemize}
\centering
\caption{Docker vs podman compat vs podman native.}
\label{porownanie}
\scriptsize
\begin{tabular}{llll}
\hline
docker & rhel8    \\ \hline
docker run & podman run \\
docker build & buildah  \\
docker foo & bar \\ \hline
\end{tabular}
\normalsize
\end{table}
\end{frame}

\begin{frame}[fragile]
	\frametitle{Instalacja narzędzi do konteneryzacji za pomocą Ansible}
%\begin{lstlisting}[language=yaml]
%---
%key: value
%---
%\end{lstlisting}
	\lstinputlisting[style=yaml,firstline=1,lastline=12]{../ansible/deploy-container-tools.yml}
\end{frame}

\begin{frame}[fragile]
	\frametitle{Instalacja narzędzi do konteneryzacji za pomocą Ansible}
\begin{center}
\includegraphics[scale=0.13]{img-ansibleinception.jpg}
\end{center}
\lstinputlisting[style=yaml,firstline=13]{../ansible/deploy-container-tools.yml}
\end{frame}

\begin{frame}[fragile]
	\frametitle{Budowanie kontenera za pomocą buildah}
\lstinputlisting[style=bash]{../buildah-example/build-container.sh}
\end{frame}

\begin{frame}
\frametitle{Koniec\ldots}
\begin{center}
\includegraphics[scale=0.5]{img-oseclogo.png}

Dziękuje!

Czy są pytania?

\end{center}
\end{frame}
\end{document}

