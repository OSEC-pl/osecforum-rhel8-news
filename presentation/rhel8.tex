\documentclass[dvipsnames,table]{beamer}
\usepackage{polski}

\usetheme{Rochester}
\usecolortheme{orchid}

\usepackage{listings}
\usepackage{ucs}
\usepackage[utf8x]{inputenc}
\usepackage{wasysym}
\usepackage[normalem]{ulem}
\usepackage{amsmath}
\usepackage{hyperref}
\usepackage{tikzsymbols}

\setbeamertemplate{navigation symbols}{}
\setbeamertemplate{caption}[numbered]
\setbeamerfont{caption}{size=\scriptsize}
\setbeamercolor{framenote}{bg=OSEC-red!25}
\setbeamercolor{rednote}{bg=Red!25}
\setbeamercolor{palette primary}{use=structure,fg=white,bg=OSEC-red}
\setbeamercolor{palette secondary}{use=structure,fg=white,bg=OSEC-red2}

\setbeamertemplate{itemize item}{\scriptsize\raise1pt\hbox{\donotcoloroutermaths$\blacktriangleright$}}
\setbeamertemplate{itemize subitem}{\tiny\raise1pt\hbox{\donotcoloroutermaths$\bullet$}}
\setbeamertemplate{itemize subsubitem}{\tiny\raise1pt\hbox{\donotcoloroutermaths{--}}}

\setbeamertemplate{enumerate item}{\insertenumlabel.}
\setbeamertemplate{enumerate subitem}{\insertenumlabel.\insertsubenumlabel}
\setbeamertemplate{enumerate subsubitem}{\insertenumlabel.\insertsubenumlabel.\insertsubsubenumlabel}
\setbeamertemplate{enumerate mini template}{\insertenumlabel}

\setbeamercolor{itemize item}{fg=OSEC-red, bg=OSEC-red}
\setbeamercolor{itemize subitem}{fg=OSEC-red, bg=OSEC-red}
\setbeamercolor{itemize subsubitem}{fg=OSEC-red, bg=OSEC-red}

\setbeamercolor{section number projected}{fg=white,bg=OSEC-red}
\setbeamercolor{subsection number projected}{fg=white,bg=OSEC-red}
\setbeamercolor{button}{bg=OSEC-red,fg=white}

\setbeamertemplate{section in toc}[circle]
\setbeamertemplate{subsection in toc}[square]

\definecolor{OSEC-red}{RGB}{160,29,44}
\definecolor{OSEC-red2}{RGB}{177,76,12}
\hypersetup{colorlinks=true,linkcolor=white,urlcolor=OSEC-red}

\setlength{\tabcolsep}{8pt}
\renewcommand{\arraystretch}{1.2}

\newcommand{\tri}{$\triangleright$ }


\newcommand\YAMLcolonstyle{\color{red}\mdseries}
\newcommand\YAMLkeystyle{\color{black}\bfseries}
\newcommand\YAMLvaluestyle{\color{black}\mdseries}

\makeatletter

\newcommand\language@yaml{yaml}

\expandafter\expandafter\expandafter\lstdefinelanguage
\expandafter{\language@yaml}
{
  keywords={true,false,null,y,n},
  keywordstyle=\color{darkgray}\bfseries,
  basicstyle=\YAMLkeystyle,                                 % assuming a key comes first
  sensitive=false,
  comment=[l]{\#},
  morecomment=[s]{/*}{*/},
  commentstyle=\color{purple}\ttfamily,
  stringstyle=\YAMLvaluestyle\ttfamily,
  moredelim=[l][\color{orange}]{\&},
  moredelim=[l][\color{magenta}]{*},
  moredelim=**[il][\YAMLcolonstyle{:}\YAMLvaluestyle]{:},   % switch to value style at :
  morestring=[b]',
  morestring=[b]",
  literate =    {---}{{\ProcessThreeDashes}}3
                {>}{{\textcolor{red}\textgreater}}1     
                {|}{{\textcolor{red}\textbar}}1 
                {\ -\ }{{\mdseries\ -\ }}3,
}

% switch to key style at EOL
\lst@AddToHook{EveryLine}{\ifx\lst@language\language@yaml\YAMLkeystyle\fi}
\makeatother

\newcommand\ProcessThreeDashes{\llap{\color{cyan}\mdseries-{-}-}}

\lstdefinestyle{yaml}
{
   language=yaml,
   basicstyle=\small\ttfamily,
   breaklines=true,
   escapechar=\@,
}
\lstdefinestyle{bash}
{
   language=bash,
   basicstyle=\tiny\ttfamily,
   breaklines=true,
   escapechar=\@,
}

\title{Nowości w Red Hat Enterprise Linux 8}
\author{Radosław Kujawa -- radoslaw.kujawa@osec.pl}
\institute{OSEC}

\begin{document}

\begin{frame}
	\titlepage
\end{frame}

\begin{frame}
\frametitle{RHEL 8 już dostępny}
\begin{center}
\includegraphics[scale=0.1]{img-rhlogo.png}
\end{center}
\begin{itemize}
	\item 8 Maja 2019 -- wydanie Red Hat Enterprise Linux 8
	\item ,,Cloud-native OS''
	\item ,,Brinding hybrid cloud to DevOps''
	\item Co zmienia się od strony technicznej?
	\item Ogromna ilość zmian -- część backportowana do RHEL 7
	\item Zarządzanie konfiguracją RHEL\ldots
	\begin{itemize}
		\item Klasyczna Unixowa linia komend
		\item Interfejs web (cockpit)
		\item Ansible -- sukcesywne wdrażanie {\em Infrastructure as a Code}
	\end{itemize}
\end{itemize}
\begin{center}
\end{center}
\end{frame}

\begin{frame}
\frametitle{Cockpit v185}
\begin{itemize}
	\item Zaawansowany interfejs webowy oparty na pluginach.
	\item Zmiany wykonane za pomocą cockpita lądują we właściwych miejscach w systemie
	\item Zarządzanie przez DBus, API, biblioteki -- cockpit unika opakowywania poleceń powłoki
	\item Standardowo włączony w RHEL 8 (poza instalacją minimal)
	\item Integracja z Red Hat IdM (IPA)
	\item {\tt dnf install cockpit\textbackslash* }
	\item {\tt systemctl enable --now cockpit.socket}
\end{itemize}
\begin{center}
\end{center}
\end{frame}

\begin{frame}
\frametitle{YUM v4 (dnf)}
\begin{itemize}
	\item Kompatybilność wsteczna z klasycznym {\tt yum}
	\item Modularyzacja systemu
	\item {\em Application Streams}
	\item Dostarczanie wielu niezależnych od siebie wersji oprogramowania w ramach jednej dystrybucji OS
	\item Zachowanie kompatybilności na poziomie API i ABI w ramach konkretnej wersji strumienia
\end{itemize}
\begin{center}
\end{center}
\end{frame}

\begin{frame}
\frametitle{YUM v4 (dnf)}
\begin{itemize}
	\item Usługi dostępne w alternatywnych wersjach
	\item Np. PostgreSQL 9.6 oraz 10
	\item Np. MariaDB 10.3 oraz MySQL 8
	\item Modularyzacja dotyczy nie tylko usług ale też języków programowania i frameworków
	\item Pojawią się strumienie z nowszymi wersjami paczek w ramach życia RHEL 8 (np. php)
	\item {\tt dnf module enable postgresql:10}
	\item {\tt dnf module install postgresql:10}
\end{itemize}
\begin{center}
\end{center}
\end{frame}



% lorax-composer

% nftables, firewalld
% ipvlan w kontenerach


% wayland! gnom 3.28, fallback do X11

% v6 only boot

% authselect, authconfig plz die
%

\begin{frame}[fragile]
	\frametitle{Nowe narzędzia do konteneryzacji}
\begin{itemize}
	\item RHEL 8 nie dostarcza Dockera
	\item Dla instalacji lokalnych, developmentu: podman
	\item Dla dużej skali (Kubernetes, OpenShift): CRI-O
	\item Red Hat Universal Base Images (UBI)
\end{itemize}
\end{frame}

\begin{frame}
	\frametitle{podman}
	\begin{itemize}
		\item Nowoczesne narzędzie do zarządznia lokalnymi kontenerami
		\item W pełni wspierany, to nie jest {\em tech preview}!
		\item Prostota konfiguracji, brak daemonów
		\item Kontenery działające na kontach użytkowników nieuprzywilejowanych
		\item Dobra integracja z systemd
		\item Kompatybilność wsteczna z Dockerem ({\tt podman-docker})
		\item Ułatwienie migracji do Kubernetesa ({\tt podman generate kube, podman play kube})
	\end{itemize}
\centering
	\begin{table}
\caption{Docker vs podman compat vs podman native.}
\label{porownanie}
\scriptsize
\begin{tabular}{llll}
\hline
docker & rhel8    \\ \hline
docker run & podman run \\
docker build & buildah  \\
docker foo & bar \\ \hline
\end{tabular}
\normalsize
\end{table}
\end{frame}

\begin{frame}[fragile]
	\frametitle{Instalacja narzędzi do konteneryzacji za pomocą Ansible}
%\begin{lstlisting}[language=yaml]
%---
%key: value
%---
%\end{lstlisting}
	\lstinputlisting[style=yaml,firstline=1,lastline=12]{../ansible/deploy-container-tools.yml}
\end{frame}

\begin{frame}[fragile]
	\frametitle{Instalacja narzędzi do konteneryzacji za pomocą Ansible}
\begin{center}
\includegraphics[scale=0.13]{img-ansibleinception.jpg}
\end{center}
\lstinputlisting[style=yaml,firstline=13]{../ansible/deploy-container-tools.yml}
\end{frame}

\begin{frame}[fragile]
	\frametitle{Budowanie kontenera za pomocą buildah}
\lstinputlisting[style=bash]{../buildah-example/build-container.sh}
\end{frame}

\begin{frame}[fragile]
	\frametitle{Aktualizacja stosu graficznego}
\begin{itemize}
	\item{Wayland domyślnym serwerem dla aplikacji graficznych}
	\item{Powrót do X11 w dalszym ciągu obsługiwany}
	\item{GNOME 3.28}
	\item \href{http://www.redhat.com}{Red Hat}
\end{itemize}
\end{frame}

\begin{frame}
\frametitle{Koniec\ldots}
\begin{center}
\includegraphics[scale=0.5]{img-oseclogo.png}

Dziękuje!

Czy są pytania?

\end{center}
\end{frame}
\end{document}

